\documentclass[a4paper,12pt]{book}
\usepackage{fullpage}
\usepackage{setspace}
\usepackage{multicol}
\usepackage{amsthm}
\usepackage{tikz}
\newcommand*\circled[1]{\tikz[baseline=(char.base)]{
            \node[shape=circle,draw,inner sep=2pt] (char) {#1};}}
\pagestyle{empty}

\theoremstyle{defn}
\newtheorem{defn}{Definition}[section]

\begin{document} 


\chapter{Linear Equations in Linear Algebra}
\section{Systems of Linear Equations}



\begin{defn}\textup{A Linear Equations is the variables $x_1,x_2...x_n$ is an equation that can be written in the form $a_1x_1+a_2x_2+...+a_nx_n = b$ where $a_1,a_2, ..., a_n$ are real coefficient and $b$ is a real number(and known)} \end{defn}
\begin{defn}\textup{A System of Linear Equations
$\left\{ \begin{array}{c}
a_{11}x_1+a_{12}x_2+...+a_{1n}x_n = b_1\\
a_{21}x_1+a_{22}x_2+...+a_{2n}x_n = b_2\\
\vdots\\
a_{m1}x_1+a_{m2}x_2+...+a_{mn}x_n = b_m
\end{array} \right.$ m number of equations, n number of unknowns (standard form) (first index row number, second index col number)}\end{defn}
\begin{defn}\textup{A solution of the system is a list ($s_1,s_2,...,s_n$) of numbers that makes each equation a true statment when the values are substituted for $x_1,x_2,...,x_n$}\end{defn}
\begin{defn}\textup{Solution Set is the set of all possible solutions}\end{defn}

Geometric Interpretations
Example) Find the Solution set of the system \\
(a) $\left\{ \begin{array}{rcl}
x_1 - x_2 & = & 5 \\
2x_1 + x_2 & = & 7
\end{array} \right.$\\
(b) $\left\{ \begin{array}{rcl}
x_1 - 2x_2&=& 4 \\
-2x_1 + 4x_2 &=& -8
\end{array} \right.$\\
(c) $\left\{ \begin{array}{rcl}
x_1+3x_2&=&1\\
2x_1+6x_2&=&5
\end{array} \right.$


\begin{defn}\textup{A linear system is consistant if it has either one solution or infinitely many solutions}\end{defn}

\begin{defn}\textup{Matrix of Coefficients} 
$\left[ \begin{array}{cccc}
a_{11} & a_{12} & \ldots & a_{1n}\\
a_{11} & a_{12} & \ldots & a_{1n}\\
\vdots &\vdots&\ddots& \vdots\\
a_{11} & a_{12} & \ldots & a_{1n}\\
\end{array} \right]$ \end{defn}


\begin{defn}\textup{Augmented Matrix of the System}
$\left[ \begin{array}{cccc|c}
a_{11} & a_{12} & \ldots & a_{1n} & b_1\\
a_{21} & a_{22} & \ldots & a_{2n} & b_2\\
\vdots &\vdots&\ddots& \vdots &\vdots\\
a_{m1} & a_{m2} & \ldots & a_{mn} & b_m\\
\end{array}\right]$ \end{defn}


\section{Row Reduction and Echelon Forms}
\begin{defn}\textup{A leading  of a row in a matrix is the left most non-zero entry} \end{defn}
Example) $\left[ \begin{array}{cccccc} 0 & 0 & \circled{7} & 3 & 4 & 1\\ \circled{2} & 4 & 0 &0 &0 &0 \\ 0 &0 &0 &0 &\circled{-2} &0 \end{array} \right] $

\begin{defn}\textup{A rectangular matrix is in echelon form if it has the following three properties:
\begin{enumerate}
\item All non-zero rows are above any zero rows.
\item Each leading entry of a row is in a column to the right of the leading entry above it.
\item All entries in a column below a leading entry are zero.
\end{enumerate}} \end{defn}

\section{Vector Equations}
\begin{defn}
\textup{Vectors\\
In $R^2, \vec{v} = \left[\begin{array}{c} v_1 \\ v_2 \end{array}\right]$, in $R^3, \vec{v}=\left[\begin{array}{c} v_1 \\ v_2 \\v_3\end{array}\right]$, in $R^n, \vec{v} =\left[\begin{array}{c} v_1 \\ v_2 \\ \vdots \\ v_n\end{array}\right]$}\end{defn}
\begin{defn}
\textup{
Alebraic Operations of Vectors.\\
$\vec{u}=\left[\begin{array}{c} u_1\\u_2\\u_3 \end{array}\right]$
$\vec{v}=\left[\begin{array}{c} v_1\\v_2\\v_3 \end{array}\right]$\\
Addition:
}\end{defn}
\end{document}